\documentclass[hidelinks, 12pt, a4paper]{article}

\usepackage[utf8]{inputenc}
\usepackage[margin=1.5cm]{geometry}
\usepackage{graphicx}
\usepackage{setspace}
\usepackage[T1]{fontenc}
\usepackage{tocloft}
\usepackage{todonotes}
\usepackage{epstopdf} 
\usepackage{hyperref}
\usepackage{float}
\usepackage{titlesec}
\usepackage{listings}
\usepackage{multirow}
\usepackage{xcolor}
\usepackage{mwe}
\usepackage{hyperref}
\onehalfspacing
\usepackage[english]{babel}
\usepackage{fancyhdr}
\usepackage{enumitem}

\pagestyle{fancy}
\fancyhf{}
\rhead{blulancetech@gmail.com}
\lhead{Carpool}
\rfoot{Page \thepage}

\author{}
\date{}
\title
{
	\includegraphics[width=6cm]{images/up_logo.jpg} \\
	Department of Computer Science \\
	Faculty of Engineering, Built Environment \& IT\\
	University of Pretoria \\
	\vspace{0.5cm}
	\Huge COS301 -
	Software Engineering\\
	\vspace{1cm}
	{\Huge Carpool}\\
	\begin{Large}
	Carpool Coding standards document
	\end{Large}
	\vspace{0.5cm}
	
    \begin{center}
    \noindent
    \includegraphics[width=6cm]{images/company_logo.png} 
    \vspace{0.5cm}
    \begin{table}[h]
    \centering
    \begin{tabular}{|l|l|l|}
    \hline
    Name  & Student Number\\ \hline
    Benjamin Osmers & u16068344 \\ \hline
    Ashleigh Govender &  U20528834      \\ \hline
    Joshua Brink  & U19185678 \\ \hline
    Jason Antalis     & U19141859     \\ \hline
    Wesley Pachai & U20578688    \\ \hline
            
    \end{tabular}
    \end{table}
    \end{center}
    }

\begin{document}
\maketitle


\newpage
\tableofcontents
\newpage
\section{Introduction}

    The goal of this paper is to lay out the coding guidelines for the Carpool Application. It ensures that our code has a consistent style, is clear, flexible, reliable, and efficient.
   \vspace{1cm} 
\section{Naming Conventions}

  When it comes to naming conventions, the following guidelines should be followed:
    
       \subsection{\large{\textbf{Classes:}}}
        \begin{itemize}[]
            \item Pascal Case / Camel Case...
            \item Spaces / Underscore
        \end{itemize}
        
        \vspace{0.5cm} 
        
         \subsection{\large{\textbf{Function headers and Variables:}}}
        \begin{itemize}[]
            \item  Pascal Case / Camel Case...
            \item   Did we capitalise certain variables
        \end{itemize}
        
         \vspace{0.5cm} 
        
         \subsection{\large{\textbf{Schema:}}}
        \begin{itemize}[]
            \item  How did we name the attributes?
            \item  How did we name the tables? 
            \item aliases?
        \end{itemize}
        
        \subsection{\large{\textbf{API:}}}
        \begin{itemize}[]
            \item  
            \item  
            \item 
        \end{itemize}
        
\newpage
        
\section{Repository Branching Structure}
\subsection{Branching}
\vspace{0.5cm} 

\subsection{Repository Structure}
\vspace{0.5cm} 

\subsection{Pull Requests and Commits}
\vspace{0.5cm} 

\newpage  

\section{Comments}
\newpage

\section{Technology Choices}
\subsection{Front-End}
The chosen front-end technology is React Native along side redux.
React Native allows us to create mobile applications using website technology and create cross-platform mobile applications, Android and iOS for example.
It allows faster development since the apps do not need to be recompiled but rather the app can be reloaded and the changes will be reflected on the device.
Flutter and Angular are also popular front-end technologies.
We opted for React Native since it provides accessibility to intelligent debugging tools and error reporting.
As well as, Flutter cannot create apps for android and Angular has lowered app performance when compared with React Native.
Unfortunately, React Native does not support parallel threading and multi-processing.
\newline
\newline
Looking at the candidates of redux, flux and mvc, we chose redux as it is a state management framework that is easy to use and easy to understand.
All of these candidates revolve around the Model View Controller pattern.
Rather than placing state information in multiple Stores across the application, Redux keeps everything in one region of the app.
\subsection{Back-End}
NestJS, GraphQL and the Prisma Client are all the choices for the back-end technologies.
Looking at our architecture, we chose NestJS as it is can be based on the Model View Controller pattern.
While ExpressJs does not follow the MVCPattern which our architecture is based on, it does not have as much structure as NestJs causing it to be inefficient.
On the API side, we chose GraphQL as it is a query language and REST is more of an architectural pattern/style.
Due to using the Prisma ORM, it is logical to therefore use the prisma client for our database needs.
\subsection{Database}
Database systems are used to store and retrieve data.
The two options were PostgreSQL and MySQL.
We opted for PostgreSQL since it has much more functionality than MySQL.
For example PostgreSQL allows us to create different data types for a specific array of data.
Another advantage over MySQL is that it is supports analytic functions.
\newline
\newline
Onto Object Relational Mapping (ORM) frameworks, we chose Prisma.
It makes database access easy with auto-generated query builder for TypeScript.
\subsection{Hosting}
When looking at hosting options, there many options which range from AWS, Azure and Google.
We chose Google Hosting because it is the most cost effective and it is also a popular option.
There is not a lot of difference between the various hosting services and generally comes down to preference.
\newpage

\section{File Structure}
Back-end
    
        \begin{lstlisting}

ExampleSubsystem
    |------ Controller
               |------ ExampleSubsystemApi
               |------ ExampleSubsystemApiController
    |------ Exception
    |------ Model
    |------ Repository
    |------ Request
    |------ Response
    |------ Service
               |------ ExampleSubsystemService
               |------ ExampleSubsystemServiceImpl
        \end{lstlisting}
            
        Front-end
    
        \begin{lstlisting}
        


services
    |---- ServiceName
tabs
    |---- ExampleTab
            |----- TabSubPage
        \end{lstlisting}
            
\end{document}