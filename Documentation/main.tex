\documentclass[hidelinks, 12pt, a4paper]{article}

\usepackage[utf8]{inputenc}
\usepackage[margin=1.5cm]{geometry}
\usepackage{graphicx}
\usepackage{setspace}
\usepackage[T1]{fontenc}
\usepackage{tocloft}
\usepackage{todonotes}
\usepackage{epstopdf} 
\usepackage{hyperref}
\usepackage{float}
\usepackage{titlesec}
\usepackage{listings}
\usepackage{multirow}
\usepackage{xcolor}
\usepackage{mwe}
\usepackage{hyperref}
\onehalfspacing
\usepackage[english]{babel}
\usepackage{fancyhdr}

\pagestyle{fancy}
\fancyhf{}
\rhead{blulancetech@gmail.com}
\lhead{Carpool}
\rfoot{Page \thepage}

\author{}
\date{}
\title
{
	\includegraphics[width=6cm]{images/up_logo.jpg} \\
	Department of Computer Science \\
	Faculty of Engineering, Built Environment \& IT\\
	University of Pretoria \\
	\vspace{0.5cm}
	\Huge COS301 -
	Software Engineering\\
	\vspace{1cm}
	{\Huge Carpool}\\
	\begin{Large}
		Software Requirements and Design Specifications
	\end{Large}
	\vspace{0.5cm}
	
    \begin{center}
    \noindent
    \includegraphics[width=6cm]{images/company_logo.png} 
    \vspace{0.5cm}
    \begin{table}[h]
    \centering
    \begin{tabular}{|l|l|l|}
    \hline
    Name  & Student Number\\ \hline
    Benjamin Osmers & u16068344 \\ \hline
    Ashleigh Govender &  U20528834      \\ \hline
    Joshua Brink  & U19185678 \\ \hline
    Jason Antalis     & U19141859     \\ \hline
    Wesley Pachai & U20578688    \\ \hline
            
    \end{tabular}
    \end{table}
    \end{center}
    }
% \setlength {\marginparwidth}{2cm}
\begin{document}
\maketitle


\newpage
\tableofcontents 
\newpage
\section{Project Information}
    
        \qquad \textbf{Project Name:} 
        
           \qquad  \qquad Carpool\\
        
        \textbf{Owner Contact Details:}

          \qquad  \qquad Matthew Wood \qquad \qquad \href{matthew.wood@advance.io}{matthew.wood@advance.io} \qquad \qquad
          
           \qquad  \qquad Keegan Ferrett \qquad \qquad \href{keeganf@mit.edu}{keeganf@mit.edu}\\
           
           
                   \textbf{Lecture mentor:}

          \qquad  \qquad David  \qquad \qquad \href{matthew.wood@advance.io}{matthew.wood@advance.io} \qquad \qquad
          
          	\vspace{1.5cm}
          	
\section{Introduction}
Carpooling is a method of sharing unoccupied seats in a car with people who commute along the same route. Usually, one person from the group drives their personal car and the others all share the cost of the trip. Due to the environmental crisis that the world faces, carpooling has become immensely popular as it reduces the number of cars on the road. Of recent carpooling has become the latest trend as petrol prices continue to increase. Carpooling has evolved into a viable, cost-effective, and stress-free mode of transportation. Finding individuals to carpool with is challenging, since it is difficult to locate someone travelling to the same destination at the same time as you. This Carpool application is a solution to this problem. \\ \\
Carpool is a mobile application that helps students find affordable transport to and from campus or longer trips such as returning home for semester break. Carpool provides students with a central location whereby they can post, find, and join safe car trips. This application allows students with vehicles to save on petrol costs as well as provide affordable transportation to students who do not have vehicles. Students can travel to their desired location while sharing the car and expenditures.\\ \\
\textbf{Vision:}\\
A more affordable and practical version of Uber/Lyft for students. Students can be both drivers and passengers.\\ \\
\textbf{Objective and Scope:} \\
To create an Android and IOS mobile application for students, whereby they are able to post, find and join safe car trips with ease.

\newpage
\section{Users}

\subsection{User Characteristics}
    
The user should have the ability to operate and own a mobile device such as a smartphone or tablet. The user should have constant access to the internet and location services to download and utilize the Carpool application. The following are users of the Carpool System:\\
    
    \Large{ \textbf{Admin} }
    \normalsize
        \begin{itemize}
            \item A person who manages the Carpool system
            \item A person who verifies a user’s email address and driver’s license.
            \item 	A person who manages the users
            \item 	A person who does maintenance and routine checks on the system
            \item 	A person who can help users with the functionality and navigation of the system

        \end{itemize}
    
    \vspace{0.5cm}
    \Large{\textbf{User}}
    \normalsize
    
    Students will be the main users of the application and will be verified through their university email. Students can be broken up into two categories namely Passengers and Drivers.\\
    
    \large{ \textbf{General} }
    
        \begin{itemize}

            \item A person who has access to a smart device that has location and internet services.
            \item A person that is studying at a recognised university in South Africa.
            \item A person who is able to share their locational services.
            \item A person who can communicate with other users through the chat functionality on the platform.
            \item A person who does not mind sharing their journey with other people.
            \item A person who can view other user's profile.
            \item A person who can use the map services provided by the platform.
        \end{itemize}
        \vspace{5.0cm}
        
    \large{ \textbf{Driver} }
    
        \begin{itemize}

            \item A person who is 18 years and older and has a valid driver license.
            \item A person who owns/ has access to a road worthy vehicle.
            \item A person who wants to Create Trips.
            \item A person who can decline or accept users that book seats.
            \item A person who can cancel trips that they have created.
            \item A person who is responsible for starting and ending trips.
            \item A person that can send reminders and information about the trip to users on the trip.
            \item  A person who is responsible for picking up and dropping off passengers at their location.
        \end{itemize}
        \vspace{0.5cm}
        
            \large{ \textbf{Passenger} }
    
        \begin{itemize}

            \item A person who can search and book trips.
            \item A person who owns a credit card and can afford to book a trip.
            \item A person who can rate their trip experience.
            \item A person who can decline or accept users that book seats.
            \item A person who can cancel trips that they have created.
            \item A person who is responsible for starting and ending trips.
            \item A person that can send reminders and information about the trip to users on the trip.
            \item  A person who is responsible for picking up and dropping off passengers at their location.
        \end{itemize}

    \newpage
\subsection{User Stories}
 \begin{changemargin}
    \begin{itemize}
        \item \textbf{\underline{User Story 1: }}\\
        As an Admin I want to verify a student's email address and driver license, so that only valid users can access the application (Students who study at a recognized University in South Africa).
        \item \textbf{\underline{User Story 2: }} \\
        As an Admin I want to review user's reviews so that I can prohibit users who have bad ratings
        \item \textbf{\underline{User Story 3: }} \\
        As a user I want to be able to login to my account and change my personal details on my profile such as (profile picture, name, surname, email, banking details and contact number).
        \item \textbf{\underline{User Story 4: }} \\
        As a User I want to view my previous trips so that I can have a record of all User I have been in contact with and view my previous destinations.
        \item \textbf{\underline{User Story 5: }} \\
        As a driver I want to be able to create my own trip by posting the date, seats available, price per seat, starting location and destination of the trip. This will help me find individuals to carpool with. I would also like to edit and delete said trips.
        \item \textbf{\underline{User Story 6: }}\\
       As a driver I want to be able to accept or decline passengers that can book a seat on my trip, this ensures that I am comfortable with those that are sharing the journey with me.
        \item \textbf{\underline{User Story 7: }} \\
        As a driver I want to receive income from my trips so that I can pay for the costs occurred on the trip (petrol).
        \item \textbf{\underline{User Story 8: }} \\ 
        As a driver I want to be able to send messaged to Passenger so that I can communicate with them information about the trip. I want to be able to communicate with them all at the same time and not send messages to them individually.
        \item \textbf{\underline{User Story 9: }}: \\
       As a Passenger I want to be able to view all trips that have been recommended to me based on my current location and previous trips so that I can pick the most suitable trip to join at ease.
        \item \textbf{\underline{User Story 10: }} \\
       As a Passenger I want to be able to enter my start and destination location in order to find trips that match my specification.
        \item \textbf{\underline{User Story 11: }} \\
       As a Passenger I want to be able to rate my trip experience so that future passengers can get insight on the driver and their trips.
        \item \textbf{\underline{User Story 12: }} \\
        As a Passenger I want to be able to view driver's ratings allowing me to join trips with higher ratings (small chance of something going wrong on the trip if a driver has a high rating).
        \item \textbf{\underline{User Story 13: }} \\
       As a budget-conscious driver  I'd like to spend less money on my commute so that I can pay off other monthly expenditures. 
        \item \textbf{\underline{User Story 14: }} \\
        As an Environmental activist user. I'd like to reduce the number of cars on the road so that I can play my part in reducing the world’s carbon footprint.
        \item \textbf{\underline{User Story 15: }} \\
        As a women user, I want a method of transportation that I can trust so that I do not have to worry about my safety as a woman.
        \item \textbf{\underline{User Story 16: }} \\ 
        As a student with a limited budget user, I want an inexpensive way to travel so I don't have to rely on public transportation.
        \item \textbf{\underline{User Story 17: }} \\ 
        As a commuter on a long route, I wany to be able to chat to someone so that the travel feels shorter.
        
        
    \end{itemize}
  \end{changemargin}
        
    
    

% \end{table}

\end{document}
